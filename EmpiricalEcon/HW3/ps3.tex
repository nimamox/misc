%Problem Set 2
%%%%%%%%%%%%%%%%%%%%%%%%%%%%%%%%%%%%%%%%%%%%%%%%

%Sweave('C:/Klaus/AAEC5126/problemsets/ps3',syntax=SweaveSyntaxNoweb)

%I) DEFINE DOCUMENTCLASS AND LOAD ALL REQUIRED PACKAGES
\documentclass[11pt,reqno]{amsart}\usepackage[]{graphicx}\usepackage[]{color}
% maxwidth is the original width if it is less than linewidth
% otherwise use linewidth (to make sure the graphics do not exceed the margin)
\makeatletter
\def\maxwidth{ %
  \ifdim\Gin@nat@width>\linewidth
    \linewidth
  \else
    \Gin@nat@width
  \fi
}
\makeatother

\definecolor{fgcolor}{rgb}{0.196, 0.196, 0.196}
\newcommand{\hlnum}[1]{\textcolor[rgb]{0.063,0.58,0.627}{#1}}%
\newcommand{\hlstr}[1]{\textcolor[rgb]{0.063,0.58,0.627}{#1}}%
\newcommand{\hlcom}[1]{\textcolor[rgb]{0.588,0.588,0.588}{#1}}%
\newcommand{\hlopt}[1]{\textcolor[rgb]{0.196,0.196,0.196}{#1}}%
\newcommand{\hlstd}[1]{\textcolor[rgb]{0.196,0.196,0.196}{#1}}%
\newcommand{\hlkwa}[1]{\textcolor[rgb]{0.231,0.416,0.784}{#1}}%
\newcommand{\hlkwb}[1]{\textcolor[rgb]{0.627,0,0.314}{#1}}%
\newcommand{\hlkwc}[1]{\textcolor[rgb]{0,0.631,0.314}{#1}}%
\newcommand{\hlkwd}[1]{\textcolor[rgb]{0.78,0.227,0.412}{#1}}%
\let\hlipl\hlkwb

\usepackage{framed}
\makeatletter
\newenvironment{kframe}{%
 \def\at@end@of@kframe{}%
 \ifinner\ifhmode%
  \def\at@end@of@kframe{\end{minipage}}%
  \begin{minipage}{\columnwidth}%
 \fi\fi%
 \def\FrameCommand##1{\hskip\@totalleftmargin \hskip-\fboxsep
 \colorbox{shadecolor}{##1}\hskip-\fboxsep
     % There is no \\@totalrightmargin, so:
     \hskip-\linewidth \hskip-\@totalleftmargin \hskip\columnwidth}%
 \MakeFramed {\advance\hsize-\width
   \@totalleftmargin\z@ \linewidth\hsize
   \@setminipage}}%
 {\par\unskip\endMakeFramed%
 \at@end@of@kframe}
\makeatother

\definecolor{shadecolor}{rgb}{.97, .97, .97}
\definecolor{messagecolor}{rgb}{0, 0, 0}
\definecolor{warningcolor}{rgb}{1, 0, 1}
\definecolor{errorcolor}{rgb}{1, 0, 0}
\newenvironment{knitrout}{}{} % an empty environment to be redefined in TeX

\usepackage{alltt}   %keep it simple
\usepackage{hyperref}
\usepackage{graphicx}      % for fancy graphics
\usepackage{setspace}      % for basic formatting
\usepackage{enumerate}     % for more flexibility with numbered lists
%\SweaveOpts{keep.source=TRUE}  %KEY - this preserves R formatting and comments

% You may need to load all or some of these packages -
%follow the instructions on our course web site under "Help with LaTex"

%II) PREAMBLE
%%%%%%%%%%%%%%%%%%%%%%%%%%%%%%%%%%%%%%%%%%%%%%%%%%
\pagestyle{plain} %puts page number center bottom
\setlength{\topmargin}{0in}
\setlength{\textheight}{8.5in}
\setlength{\oddsidemargin}{.0in}
\setlength{\evensidemargin}{.0in}
\setlength{\textwidth}{6.5in}
\setlength{\footskip}{.5in}
\setlength{\parindent}{0in} %suppress indentation
%\onehalfspacing

\newcommand{\mlt}[1]{\mathbf{#1}} %matrix bold for Latin symbols
\newcommand{\mgr}[1]{\boldsymbol{#1}}%matrix bold for Greek symbols
\newcommand{\kR}{\tt R\rm{} }%shortcut for "R" symbol
\newcommand{\ksp}{\vspace{0.1in}}   % insert some space between chunks
%feel free to add your own shortcuts  - here a mine:
\newcommand{\kl}{\left(}
\newcommand{\kr}{\right)}
\newcommand{\kll}{\left\{}
\newcommand{\krr}{\right\}}
\newcommand{\kmu}{\mgr{\mu}}
\newcommand{\kpsi}{\mgr{\psi}}
\newcommand{\kphi}{\mgr{\phi}}
\newcommand{\kgam}{\mgr{\gamma}}
\newcommand{\ktheta}{\mgr{\theta}}
\newcommand{\kbeta}{\mgr{\beta}}
\newcommand{\kdelta}{\mgr{\delta}}
\newcommand{\kt}{^{\prime}}
\newcommand{\kdel}{\partial}
\newcommand{\kdot}{\kl . \kr}
\newcommand{\keps}{\epsilon}
\newcommand{\kx}{\mlt{x}}
\newcommand{\kX}{\mlt{X}}
\newcommand{\kV}{\mlt{V}}
\newcommand{\kM}{\mlt{M}}
\newcommand{\kP}{\mlt{P}}
\newcommand{\ky}{\mlt{y}}
\newcommand{\kb}{\mlt{b}}
\newcommand{\kc}{\mlt{c}}
\newcommand{\ki}{\mlt{i}}
\newcommand{\ke}{\mlt{e}}
\newcommand{\klam}{\lambda}
\newcommand{\kp}{\mlt{p}}
\newcommand{\kprob}{\text{prob}}
\newcommand{\kz}{\mlt{z}}
\newcommand{\ksig}{\sigma^2}
\newcommand{\kSig}{\mgr{\Sigma}}
\newcommand{\klog}{\text{log}}
\newcommand{\kols}{\kl \kX\kt\kX\kr^{-1}\kX\kt\ky}
\newcommand{\kSSE}{\kl \ky-\kX\kb\kr\kt\kl\ky-\kX\kb\kr}

%%%%%%%%%%%%%%%%%%%%%%%%
\IfFileExists{upquote.sty}{\usepackage{upquote}}{}
\begin{document}
%\SweaveOpts{concordance=TRUE}
%%%%%%%%%%%%%%%%%%%%%%%%

%III) TOP MATTER INFORMATION
\title{Problem Set 3}
\author{Nima Mohammadi \\ \href{mailto:nimamo@vt.edu}{\textbf{nimamo@vt.edu}}} %ENTER YOUR NAME HERE
\maketitle %this comes at the end of the top matter to set it.











%%%%%%%%%%%%%%%%%%%%%%%%%%%%%%%%%%%%%%%%%%%%%%%%%%%%%%%%%%%%%%%%%%%%%%%%
\section{MLE Bernoulli}
%%%%%%%%%%%%%%%%%%%%%%%%%%%%%%%%%%%%%%%%%%%%%%%%%%%%%%%%%%%%%%%%%%%%%%%%
Consider the Bernoulli model (also called "Binary response model") where a given random draw $y_i = 1$ with probability $\pi$, 
and $y_i = 0$ with probability $1-\pi$.  
Thus, the density for this model can be compactly written as $f\kl y_i \kr =\pi^{y_i}\kl 1-\pi \kr^{1-y_i}$.  
The moments for this density are given as $E\kl y_i \kr = \pi, V\kl y_i \kr =\pi\kl 1-\pi \kr$. 
Assume you draw a sample of $n$ observations from this distribution.
\ksp
\begin{enumerate}
\item  Derive $lnL\kl \pi \kr, g\kl \pi \kr, H\kl \pi \kr, I\kl \pi \kr$.

$f\kl y_i \kr =\pi^{y_i}\kl 1-\pi \kr^{1-y_i}$\\
$\Rightarrow l\kl \pi \kr =\pi^{y_i}\kl 1-\pi \kr^{1-y_i}$\\
$\Rightarrow \ln l\kl \pi \kr = y_i \ln \kl \pi \kr + \kl 1-y_i\kr \ln\kl 1-\pi \kr$\\

Then, summing over all observations we have:
\ksp

\begin{equation*}
\begin{split}
\ln \mathrm{L}(\pi)&=\ln (\pi) \sum_{i=1}^{n} y_{i}+\ln (1-\pi) \sum_{i=1}^{n}\left(1-y_{i}\right)\\
g(\pi)&=\frac{\partial \ln L(\pi)}{\partial \pi}=\frac{1}{\pi} \sum_{i=1}^{n} y_{i}-\frac{1}{1-\pi} \sum_{i=1}^{n}\left(1-y_{i}\right)\\
H(\pi)&=\frac{\partial^{2} \ln L(\pi)}{\partial \pi^{2}}=-\frac{1}{\pi^{2}} \sum_{i=1}^{n} y_{i}-\frac{1}{(1-\pi)^{2}} \sum_{i=1}^{n}\left(1-y_{i}\right)\\
I(\pi)&=-\mathrm{E}_{y}(H(\pi))=-\mathrm{E}_{y}\left(-\frac{1}{\pi^{2}} \sum_{i=1}^{n} y_{i}-\frac{1}{(1-\pi)^{2}} \sum_{i=1}^{n}\left(1-y_{i}\right)\right)\\&=\frac{n \pi}{\pi^{2}}+\frac{n(1-\pi)}{(1-\pi)^{2}}=\frac{n}{\pi(1-\pi)}
\end{split}
\end{equation*}



\item	Derive the ML estimator (call it $p$), and its asymptotic variance.

\begin{equation*}
\begin{split}
& g(\pi)=\frac{1}{\pi} \sum_{i=1}^{n} y_{i}-\frac{1}{1-\pi} \sum_{i=1}^{n}\left(1-y_{i}\right)=0\\
& \Rightarrow \frac{1}{\pi} \sum_{i=1}^{n} y_{i}=\frac{1}{1-\pi} \sum_{i=1}^{n}\left(1-y_{i}\right) \Rightarrow \frac{1}{\pi} \sum_{i=1}^{n} y_{i}=\frac{n}{1-\pi}-\frac{1}{1-\pi} \sum_{i=1}^{n} y_{i} \\
& \Rightarrow \frac{1}{\pi} \sum_{i=1}^{n} y_{i}+\frac{1}{1-\pi} \sum_{i=1}^{n} y_{i}=\frac{n}{1-\pi} \Rightarrow \left(\frac{1}{\pi}+\frac{1}{1-\pi}\right) \sum_{i=1}^{n} y_{i}=\frac{n}{1-\pi}\\
&\Rightarrow \sum_{i=1}^{n} y_{i} = n\pi\\
&\Rightarrow p=\frac{\sum_{i=1}^{n} y_{i}}{n}
\end{split}
\end{equation*}

$Var(p) =[I(\pi)]^{-1} =[\dfrac{n}{\pi(1-\pi)}]^{-1} = \dfrac{\pi(1-\pi)}{n}$

\item	Given a sample of four "1"'s and one "0" (so n=5), compute $lnL\kl \pi \kr$ in terms of $\pi$ and numerically.

Our estimator gives
$$p=\frac{\sum_{i=1}^{n} y_{i}}{n}=\frac{4}{5}=0.8$$

We have
\begin{equation*}
\begin{split}
f(y_i = 1) &= \pi\\
f(y_i = 0) &= (1 - \pi)
\end{split}
\end{equation*}

Then the likelihood function is evaluated as
\begin{equation*}
\begin{split}
L(\pi)=\pi^{4}(1-\pi) \Rightarrow \ln L(\pi)=4 \ln (\pi)+\ln (1-\pi)
\end{split}
\end{equation*}

Then

$$\ln L(p)=4 \ln (.8)+\ln (1-.8)=-2.502$$

\item	Now suppose that in the sample of 5 observations, all are "1's".  
Derive the numerical solution for the ML estimator for this case. 

\begin{equation*}
\begin{split}
& p=\frac{\sum_{i=1}^{n} y_{i}}{n}=\frac{5}{5}=1\\
\Rightarrow & \ln L(\pi)=5 \ln (\pi)\\
\Rightarrow & \ln L(p)=5 \ln (1)=0
\end{split}
\end{equation*}

\end{enumerate}

%%%%%%%%%%%%%%%%%%%%%%%%%%%%%%%%%%%%%%%%%%%%%%%%%%%%%%%%%%%%%%%%%%%%%%%%
\section{MLE Parameterized Exponential}
%%%%%%%%%%%%%%%%%%%%%%%%%%%%%%%%%%%%%%%%%%%%%%%%%%%%%%%%%%%%%%%%%%%%%%%%
Consider the Exponential density with parameterized mean, i.e
\begin{equation*}
\begin{split}
&f\kl y_i |\mlt{x}_i\kr = \frac{exp\kl -y_i /\mlt{x}_i\kt\mgr{\beta}\kr}{\mlt{x}_i\kt\mgr{\beta}} \quad \text{with} \\
&E\kl y_i |\mlt{x}_i\kr= \mlt{x}_i\kt\mgr{\beta}, \quad V\kl y_i|\mlt{x}_i\kr = \kl  \mlt{x}_i\kt\mgr{\beta}\kr^2
\end{split}
\end{equation*}
\ksp

Assume the sample size is $n$ and $\mlt{x}_i$ is $k \times 1$.
\ksp

\begin{enumerate}
\item Derive $lnL\kl \mgr{\beta}\kr, g \kl \mgr{\beta}\kr,H\kl \mgr{\beta}\kr$ and $I\kl \mgr{\beta}\kr$

$$l(\boldsymbol{\beta})=\frac{\exp \left(-y_{i} / \mathbf{x}_{i}^{\prime} \boldsymbol{\beta}\right)}{\mathbf{x}_{i}^{\prime} \boldsymbol{\beta}}$$\\
$$\ln l(\boldsymbol{\beta})=-\frac{y_{i}}{\mathbf{x}_{i}^{\prime} \boldsymbol{\beta}}-\ln \left(\mathbf{x}_{i}^{\prime} \boldsymbol{\beta}\right)$$

\begin{equation*}
\begin{split}
\ln L(\boldsymbol{\beta})&=-\sum_{i=1}^{n} y_{i}\left(\mathbf{x}_{i}^{\prime} \boldsymbol{\beta}\right)^{-1}-\sum_{i=1}^{n} \ln \left(\mathbf{x}_{i}^{\prime} \boldsymbol{\beta}\right)\\
g(\boldsymbol{\beta})&=\frac{\partial \ln L(\boldsymbol{\beta})}{\partial \boldsymbol{\beta}}=\sum_{i=1}^{n}\left[y_{i}\left(\mathbf{x}_{i}^{\prime} \boldsymbol{\beta}\right)^{-2} \mathbf{x}_{i}\right]-\sum_{i=1}^{n}\left(\mathbf{x}_{i}^{\prime} \boldsymbol{\beta}\right)^{-1} \mathbf{x}_{\mathbf{i}}\\
H(\boldsymbol{\beta})&=\frac{\partial g(\boldsymbol{\beta})}{\partial \boldsymbol{\beta}^{\prime}}=-2 \sum_{i=1}^{n}\left[y_{i}\left(\mathbf{x}_{i}^{\prime} \boldsymbol{\beta}\right)^{-3} \mathbf{x}_{i} \mathbf{x}_{i}^{\prime}\right]+\sum_{i=1}^{n}\left(\mathbf{x}_{i}^{\prime} \boldsymbol{\beta}\right)^{-2} \mathbf{x}_{i} \mathbf{x}_{i}^{\prime}\\
I(\boldsymbol{\beta})&=-E_{y}[H(\boldsymbol{\beta})]=2 \sum_{i=1}^{n} E\left(y_{i}\right)\left(\mathbf{x}_{i}^{\prime} \boldsymbol{\beta}\right)^{-3} \mathbf{x}_{i} \mathbf{x}_{i}^{\prime}-\sum_{i=1}^{n}\left(\mathbf{x}_{i}^{\prime} \boldsymbol{\beta}\right)^{-2} \mathbf{x}_{i} \mathbf{x}_{i}^{\prime}\\
&=2 \sum_{i=1}^{n}\left(\mathbf{x}_{i}^{\prime} \boldsymbol{\beta}\right)\left(\mathbf{x}_{i}^{\prime} \boldsymbol{\beta}\right)^{-3} \mathbf{x}_{i} \mathbf{x}_{i}^{\prime}-\sum_{i=1}^{n}\left(\mathbf{x}_{i}^{\prime} \boldsymbol{\beta}\right)^{-2} \mathbf{x}_{i} \mathbf{x}_{i}^{\prime}
=2 \sum_{i=1}^{n}\left(\mathbf{x}_{i}^{\prime} \boldsymbol{\beta}\right)^{-2} \mathbf{x}_{i} \mathbf{x}_{i}^{\prime}-\sum_{i=1}^{n}\left(\mathbf{x}_{i}^{\prime} \boldsymbol{\beta}\right)^{-2} \mathbf{x}_{i} \mathbf{x}_{i}^{\prime}\\
& = \sum_{i=1}^{n}\left(\mathbf{x}_{i}^{\prime} \boldsymbol{\beta}\right)^{-2} \mathbf{x}_{i} \mathbf{x}_{i}^{\prime}
\end{split}
\end{equation*}

\item Using the gradient for the entire sample, show that the score identity holds.

\begin{equation*}
\begin{split}
E_{y}[g(\boldsymbol{\beta})]&=\sum_{i=1}^{n}\left[E\left(y_{i} | \mathbf{x}_{i}\right)\left(\mathbf{x}_{i}^{\prime} \boldsymbol{\beta}\right)^{-2} \mathbf{x}_{i}-\left(\mathbf{x}_{i}^{\prime} \boldsymbol{\beta}\right)^{-1} \mathbf{x}_{\mathbf{i}}\right] \\
&= \sum_{i=1}^{n}\left[\left(\mathbf{x}_{i}^{\prime} \boldsymbol{\beta}\right)^{-1} \mathbf{x}_{i}-\left(\mathbf{x}_{i}^{\prime} \boldsymbol{\beta}\right)^{-1} \mathbf{x}_{i}\right]=0
\end{split}
\end{equation*}

\item Using the gradient for the entire sample, show that the information matrix identity holds.

\begin{equation*}
\begin{split}
V[g(\boldsymbol{\beta}) | \mathbf{X}]&=\sum_{\Sigma} V\left(y_{i} | \mathbf{x}_{i}\right)\left(\mathbf{x}_{i}^{\prime} \boldsymbol{\beta}\right)^{-4} \mathbf{x}_{i} \mathbf{x}_{i}^{\prime}\\
&=\sum_{i=1}^{n}\left(\mathbf{x}_{i}^{\prime} \boldsymbol{\beta}\right)^{2}\left(\mathbf{x}_{i}^{\prime} \boldsymbol{\beta}\right)^{-4} \mathbf{x}_{i} \mathbf{x}_{i}^{\prime} = \sum_{i=1}^{n}\left(\mathbf{x}_{i}^{\prime} \boldsymbol{\beta}\right)^{-2} \mathbf{x}_{i} \mathbf{x}_{i}^{\prime}\\& = I(\boldsymbol{\beta})
\end{split}
\end{equation*}
\end{enumerate}

%%%%%%%%%%%%%%%%%%%%%%%%%%%%%%%%%%%%%%%%%%%%%%%%%%%%%%%%%%%%%%%%%%%%%%%%
\section{Hypothesis Testing in MLE}
%%%%%%%%%%%%%%%%%%%%%%%%%%%%%%%%%%%%%%%%%%%%%%%%%%%%%%%%%%%%%%%%%%%%%%%%
Consider the hedonic property value data set from script \texttt{mod3s2c} and 
the log-linear regression version of this model.  Estimate this model via MLE,  
using analytical gradient and Hessian (as in \texttt{mod2s1b}). 
Use the inverted negative Hessian to derive standard errors for all estimates. 
Capture your MLE results in a nice table.\\

Loading data:
\begin{knitrout}
\definecolor{shadecolor}{rgb}{1, 1, 1}\color{fgcolor}\begin{kframe}
\begin{alltt}
\hlstd{data} \hlkwb{<-} \hlkwd{read.table}\hlstd{(}\hlstr{'/Users/nima/AAEC5126/data/hedonics.txt'}\hlstd{,} \hlkwc{sep}\hlstd{=}\hlstr{"\textbackslash{}t"}\hlstd{,} \hlkwc{header}\hlstd{=}\hlnum{FALSE}\hlstd{)}

\hlkwd{colnames}\hlstd{(data)} \hlkwb{<-} \hlkwd{c}\hlstd{(}\hlstr{"price"}\hlstd{,} \hlstr{"lnacres"}\hlstd{,} \hlstr{"lnsqft"}\hlstd{,} \hlstr{"age"}\hlstd{,}\hlstr{"gradeab"}\hlstd{,} \hlstr{"pkadeq"}\hlstd{,}
                   \hlstr{"vacant"}\hlstd{,} \hlstr{"empden"}\hlstd{,} \hlstr{"popden"}\hlstd{,} \hlstr{"metro"}\hlstd{,} \hlstr{"distair"}\hlstd{,} \hlstr{"disthaz"}\hlstd{)}

\hlkwd{attach}\hlstd{(data)}
\hlstd{n} \hlkwb{<-} \hlkwd{nrow}\hlstd{(data)}
\hlstd{X} \hlkwb{<-} \hlkwd{cbind}\hlstd{(}\hlkwd{rep}\hlstd{(}\hlnum{1}\hlstd{, n), data[,}\hlopt{-}\hlnum{1}\hlstd{])}
\hlkwd{colnames}\hlstd{(X)[}\hlnum{1}\hlstd{]} \hlkwb{<-} \hlstr{"const"}
\hlstd{X} \hlkwb{<-} \hlkwd{as.matrix}\hlstd{(X)}
\hlstd{k} \hlkwb{<-} \hlkwd{ncol}\hlstd{(X)}
\hlstd{y} \hlkwb{<-} \hlkwd{log}\hlstd{(data}\hlopt{$}\hlstd{price)}
\end{alltt}
\end{kframe}
\end{knitrout}

\ksp

Function for optimization components (llf, g, H):

\begin{knitrout}
\definecolor{shadecolor}{rgb}{1, 1, 1}\color{fgcolor}\begin{kframe}
\begin{alltt}
\hlstd{CLRMllfan} \hlkwb{<-} \hlkwa{function}\hlstd{(}\hlkwc{x}\hlstd{,} \hlkwc{y}\hlstd{,} \hlkwc{X}\hlstd{,} \hlkwc{n}\hlstd{,} \hlkwc{k}\hlstd{)\{}
    \hlstd{bm} \hlkwb{<-} \hlstd{x[}\hlnum{1}\hlopt{:}\hlstd{k]}
    \hlstd{sig2} \hlkwb{<-} \hlstd{x[k} \hlopt{+} \hlnum{1}\hlstd{]}\hlopt{^}\hlnum{2}  \hlcom{#square to keep positive}

    \hlstd{llf} \hlkwb{<-} \hlopt{-}\hlstd{(n}\hlopt{/}\hlnum{2}\hlstd{)} \hlopt{*} \hlkwd{log}\hlstd{(}\hlnum{2} \hlopt{*} \hlstd{pi)} \hlopt{-} \hlstd{(n}\hlopt{/}\hlnum{2}\hlstd{)} \hlopt{*} \hlkwd{log}\hlstd{(sig2)} \hlopt{-} \hlstd{((}\hlnum{1}\hlopt{/}\hlstd{(}\hlnum{2} \hlopt{*}
        \hlstd{sig2))} \hlopt{*} \hlkwd{t}\hlstd{(y} \hlopt{-} \hlstd{X} \hlopt \hlstd{bm)} \hlopt \hlstd{(y} \hlopt{-} \hlstd{X} \hlopt \hlstd{bm))}
    \hlcom{# sample log-lh for the CLRM}

    \hlcom{# Gradient}
    \hlstd{g1} \hlkwb{<-} \hlstd{(}\hlkwd{t}\hlstd{(X)} \hlopt \hlstd{(y} \hlopt{-} \hlstd{X} \hlopt \hlstd{bm))}\hlopt{/}\hlstd{sig2}
    \hlstd{g2} \hlkwb{<-} \hlopt{-}\hlstd{(n}\hlopt{/}\hlstd{(}\hlnum{2} \hlopt{*} \hlstd{sig2))} \hlopt{+} \hlstd{((}\hlkwd{t}\hlstd{(y} \hlopt{-} \hlstd{X} \hlopt \hlstd{bm)} \hlopt \hlstd{(y} \hlopt{-}
        \hlstd{X} \hlopt \hlstd{bm))}\hlopt{/}\hlstd{((}\hlnum{2} \hlopt{*} \hlstd{sig2}\hlopt{^}\hlnum{2}\hlstd{)))}
    \hlstd{g} \hlkwb{<-} \hlkwd{rbind}\hlstd{(g1, g2)}

    \hlcom{# Hessian}
    \hlstd{H1} \hlkwb{<-} \hlopt{-}\hlstd{(}\hlkwd{t}\hlstd{(X)} \hlopt \hlstd{X)}\hlopt{/}\hlstd{sig2}
    \hlstd{H2} \hlkwb{<-} \hlopt{-}\hlstd{(}\hlkwd{t}\hlstd{(X)} \hlopt \hlstd{(y} \hlopt{-} \hlstd{X} \hlopt \hlstd{bm))}\hlopt{/}\hlstd{(sig2}\hlopt{^}\hlnum{2}\hlstd{)}
    \hlstd{H3} \hlkwb{<-} \hlkwd{t}\hlstd{(H2)}
    \hlstd{H4} \hlkwb{<-} \hlstd{n}\hlopt{/}\hlstd{(}\hlnum{2} \hlopt{*} \hlstd{sig2}\hlopt{^}\hlnum{2}\hlstd{)} \hlopt{-} \hlstd{(}\hlkwd{t}\hlstd{(y} \hlopt{-} \hlstd{X} \hlopt \hlstd{bm)} \hlopt \hlstd{(y} \hlopt{-} \hlstd{X} \hlopt
        \hlstd{bm)}\hlopt{/}\hlstd{sig2}\hlopt{^}\hlnum{3}\hlstd{)}
    \hlstd{H} \hlkwb{<-} \hlkwd{rbind}\hlstd{(}\hlkwd{cbind}\hlstd{(H1, H2),} \hlkwd{cbind}\hlstd{(H3, H4))}
    \hlkwd{return} \hlstd{(}\hlkwd{list}\hlstd{(llf, g, H))}
\hlstd{\}}
\end{alltt}
\end{kframe}
\end{knitrout}

\ksp

Initial values:

\begin{knitrout}
\definecolor{shadecolor}{rgb}{1, 1, 1}\color{fgcolor}\begin{kframe}
\begin{alltt}
\hlstd{bols} \hlkwb{<-} \hlkwd{solve}\hlstd{((}\hlkwd{t}\hlstd{(X))} \hlopt \hlstd{X)} \hlopt \hlstd{(}\hlkwd{t}\hlstd{(X)} \hlopt \hlstd{y)}  \hlcom{# compute OLS estimator}
\hlstd{e} \hlkwb{<-} \hlstd{y} \hlopt{-} \hlstd{X} \hlopt \hlstd{bols}  \hlcom{# Get residuals.}
\hlstd{SSR} \hlkwb{<-} \hlstd{(}\hlkwd{t}\hlstd{(e)} \hlopt \hlstd{e)}  \hlcom{#sum of squared residuals - should be minimized}
\hlstd{s2} \hlkwb{<-} \hlstd{(}\hlkwd{t}\hlstd{(e)} \hlopt \hlstd{e)}\hlopt{/}\hlstd{(n} \hlopt{-} \hlstd{k)}  \hlcom{#get the regression error (estimated variance of 'eps').}
\hlstd{Vb} \hlkwb{<-} \hlstd{s2[}\hlnum{1}\hlstd{,} \hlnum{1}\hlstd{]} \hlopt{*} \hlkwd{solve}\hlstd{((}\hlkwd{t}\hlstd{(X))} \hlopt \hlstd{X)}
\hlcom{# get the estimated variance-covariance matrix of bols}
\hlstd{se} \hlkwb{=} \hlkwd{sqrt}\hlstd{(}\hlkwd{diag}\hlstd{(Vb))}  \hlcom{# get the standard erros for your coefficients;}
\hlstd{tval} \hlkwb{=} \hlstd{bols}\hlopt{/}\hlstd{se}  \hlcom{# get your t-values.}
\hlstd{x0} \hlkwb{<-} \hlnum{0.7} \hlopt{*} \hlkwd{c}\hlstd{(bols, s2)}
\end{alltt}
\end{kframe}
\end{knitrout}

Choose the following tuner settings for the optimization algorithm:\\

\begin{knitrout}
\definecolor{shadecolor}{rgb}{1, 1, 1}\color{fgcolor}\begin{kframe}
\begin{alltt}
\hlstd{cri} \hlkwb{<-} \hlnum{10}  \hlcom{#initial setting for convergence criterion}
\hlstd{cri1} \hlkwb{<-} \hlnum{1e-04}  \hlcom{#convergence criterion}
\hlcom{# (here for the sum of the absolute values of the elements in}
\hlcom{# the gradient)}
\hlstd{maxiter} \hlkwb{<-} \hlnum{2000}  \hlcom{#max. number of allowed iterations}
\hlstd{stsz} \hlkwb{<-} \hlnum{0.1}  \hlcom{#step size, here it seems necessary to keep it on the small side}

\hlstd{b} \hlkwb{<-} \hlstd{x0}
\hlstd{jj} \hlkwb{<-} \hlnum{0}

\hlkwa{while} \hlstd{((cri}\hlopt{>}\hlstd{cri1)} \hlopt{&} \hlstd{(jj}\hlopt{<}\hlstd{maxiter))   \{}
    \hlstd{jj} \hlkwb{=} \hlstd{jj} \hlopt{+} \hlnum{1}

    \hlstd{int} \hlkwb{<-} \hlkwd{CLRMllfan}\hlstd{(b, y, X, n, k)}
    \hlstd{llf} \hlkwb{<-} \hlstd{int[[}\hlnum{1}\hlstd{]]}
    \hlstd{g} \hlkwb{<-} \hlstd{int[[}\hlnum{2}\hlstd{]]}
    \hlstd{H} \hlkwb{<-} \hlstd{int[[}\hlnum{3}\hlstd{]]}

    \hlstd{cri}\hlkwb{<-}\hlkwd{sum}\hlstd{(}\hlkwd{abs}\hlstd{(g))} \hlcom{#evaluate convergence criterion}
    \hlstd{db}\hlkwb{=}\hlkwd{solve}\hlstd{(}\hlopt{-}\hlstd{H)} \hlopt \hlstd{g;} \hlcom{#get directional vector}
    \hlstd{b}\hlkwb{<-} \hlstd{b}\hlopt{+}\hlstd{stsz}\hlopt{*}\hlstd{db;} \hlcom{#update b}

    \hlstd{iter} \hlkwb{<-} \hlkwd{c}\hlstd{(jj, llf, cri)}
    \hlkwd{print}\hlstd{(iter)}  \hlcom{#send iteration results to R's command window}

    \hlkwa{if} \hlstd{(jj} \hlopt{==} \hlstd{maxiter) \{}
        \hlstr{"Maximum number of iterations reached"}
        \hlkwa{break}
    \hlstd{\}}
\hlstd{\}} \hlcom{#end of "while"-loop}
\end{alltt}
\end{kframe}
\end{knitrout}
\ksp

Reporting the output:

\begin{knitrout}
\definecolor{shadecolor}{rgb}{1, 1, 1}\color{fgcolor}\begin{kframe}
\begin{alltt}
\hlstd{bm} \hlkwb{<-} \hlstd{b}  \hlcom{#this includes sigma}
\hlstd{sig2} \hlkwb{<-} \hlstd{bm[k} \hlopt{+} \hlnum{1}\hlstd{]}\hlopt{^}\hlnum{2}
\hlstd{sem} \hlkwb{<-} \hlkwd{sqrt}\hlstd{(}\hlkwd{diag}\hlstd{(}\hlkwd{solve}\hlstd{(}\hlopt{-}\hlstd{H)))}  \hlcom{#note here we need the negative H}
\hlstd{tm} \hlkwb{<-} \hlstd{bm}\hlopt{/}\hlstd{sem}

\hlstd{ttmle} \hlkwb{<-} \hlkwd{data.frame}\hlstd{(}\hlkwc{col1} \hlstd{=} \hlkwd{c}\hlstd{(}\hlstr{"constant"}\hlstd{,} \hlstr{"lnacres"}\hlstd{,} \hlstr{"lnsqft"}\hlstd{,}
    \hlstr{"age"}\hlstd{,} \hlstr{"gradeab"}\hlstd{,} \hlstr{"pkadeq"}\hlstd{,} \hlstr{"vacant"}\hlstd{,} \hlstr{"empden"}\hlstd{,} \hlstr{"popden"}\hlstd{,}
    \hlstr{"metro"}\hlstd{,} \hlstr{"distair"}\hlstd{,} \hlstr{"disthaz"}\hlstd{,} \hlstr{"sigma"}\hlstd{),} \hlkwc{col2} \hlstd{= bm,} \hlkwc{col3} \hlstd{= sem,}
    \hlkwc{col4} \hlstd{= tm)}
\hlkwd{colnames}\hlstd{(ttmle)} \hlkwb{<-} \hlkwd{c}\hlstd{(}\hlstr{"variable"}\hlstd{,} \hlstr{"estimate"}\hlstd{,} \hlstr{"s.e."}\hlstd{,} \hlstr{"t"}\hlstd{)}
\end{alltt}
\end{kframe}
\end{knitrout}

% latex table generated in R 3.6.2 by xtable 1.8-4 package
% Tue Mar 31 23:16:13 2020
\begin{table}[!h]
\centering
\caption{MLE output} 
\begin{tabular}{lrrr}
  \hline
variable & estimate & s.e. & t \\ 
  \hline
constant & 9.905 & 0.326 & 30.342 \\ 
  lnacres & 0.372 & 0.074 & 5.011 \\ 
  lnsqft & 0.595 & 0.076 & 7.839 \\ 
  age & 0.002 & 0.003 & 0.774 \\ 
  gradeab & 0.716 & 0.234 & 3.059 \\ 
  pkadeq & 0.025 & 0.130 & 0.193 \\ 
  vacant & -0.004 & 0.005 & -0.811 \\ 
  empden & 0.015 & 0.004 & 4.176 \\ 
  popden & -0.003 & 0.012 & -0.223 \\ 
  metro & 0.488 & 0.114 & 4.297 \\ 
  distair & 0.108 & 0.018 & 6.044 \\ 
  disthaz & 0.033 & 0.013 & 2.590 \\ 
  sigma & 0.849 & 0.051 & 16.554 \\ 
   \hline
\end{tabular}
\end{table}


The estimated error variance for the MLE model is 0.721. \\
The value of the log-likelihood function at convergence is -495.795\\


\begin{enumerate}[(1)]
\item Which coefficients are significant at the 1\% or 5\% level?

The variables \texttt{lnacres}, \texttt{lnsqft}, \texttt{gradeab}, \texttt{empden}, \texttt{metro}, \texttt{distair}, and \texttt{disthaz} are significant at the 1\% and 5\% levels.\\

\item Interpret the marginal effects of "lnacres" , "gradeab", and "disthaz" on the dependent variable.

{\bf"lnacres"}: When acreage of property is increased by 1 unite, sales price increases by 0.372 unite.\\

{\bf"gradeab"}: exp(.716)-1 = 1.046, so sales praice is 104.6\% more if property received the highest score from tax assessor rather than if it did not receive such a score.\\

{\bf"disthaz"}: If the distance to hazardous waste site gets increased by 1 unit (miles), sales price will increase by 3.3\%.\\

\item Derive an estimate for the \emph{variance} of the error term, along with a 95\% confidence interval.

\begin{knitrout}
\definecolor{shadecolor}{rgb}{1, 1, 1}\color{fgcolor}\begin{kframe}
\begin{alltt}
\hlstd{sig} \hlkwb{<-} \hlstd{b[k} \hlopt{+} \hlnum{1}\hlstd{]}
\hlstd{H1} \hlkwb{<-} \hlkwd{solve}\hlstd{(}\hlopt{-}\hlstd{H)}
\hlstd{Vsig} \hlkwb{<-} \hlstd{H1[k} \hlopt{+} \hlnum{1}\hlstd{, k} \hlopt{+} \hlnum{1}\hlstd{]}

\hlstd{sig2} \hlkwb{<-} \hlstd{sig}\hlopt{^}\hlnum{2}
\hlstd{Vsig2} \hlkwb{<-} \hlnum{4} \hlopt{*} \hlstd{sig}\hlopt{^}\hlnum{2} \hlopt{*} \hlstd{Vsig}
\hlstd{sesig2} \hlkwb{<-} \hlkwd{sqrt}\hlstd{(Vsig2)}

\hlcom{# Confidence interval:}
\hlstd{lo} \hlkwb{<-} \hlstd{sig2} \hlopt{-} \hlnum{1.96} \hlopt{*} \hlstd{sesig2}
\hlstd{hi} \hlkwb{<-} \hlstd{sig2} \hlopt{+} \hlnum{1.96} \hlopt{*} \hlstd{sesig2}

\hlstd{ttDELTA} \hlkwb{<-} \hlkwd{data.frame}\hlstd{(}\hlkwc{col1} \hlstd{=} \hlstr{"error variance"}\hlstd{,} \hlkwc{col2} \hlstd{= sig2,} \hlkwc{col3} \hlstd{= sesig2,}
    \hlkwc{col4} \hlstd{= lo,} \hlkwc{col5} \hlstd{= hi)}
\hlkwd{colnames}\hlstd{(ttDELTA)} \hlkwb{<-} \hlkwd{c}\hlstd{(}\hlstr{"variable"}\hlstd{,} \hlstr{"estimate"}\hlstd{,} \hlstr{"s.e."}\hlstd{,} \hlstr{"lower"}\hlstd{,}
    \hlstr{"upper"}\hlstd{)}
\end{alltt}
\end{kframe}
\end{knitrout}
\ksp

% latex table generated in R 3.6.2 by xtable 1.8-4 package
% Tue Mar 31 23:28:21 2020
\begin{table}[!h]
\centering
\caption{MLE results for error variance using DELTA method} 
\begin{tabular}{lrrrr}
  \hline
variable & estimate & s.e. & lower & upper \\ 
  \hline
error variance & 0.721 & 0.087 & 0.550 & 0.891 \\ 
   \hline
\end{tabular}
\end{table}

\ksp

\item Perform Wald tests for the following hypotheses. For each hypothesis obtain the test statistic and the corresponding p-value, and state your decision (use script \texttt{mod3s2b} for guidance on Wald tests in an MLE context):\\

    \begin{enumerate} [(a)]
    \item The marginal effects of \texttt{age} ($\beta_4$), \texttt{pkadeq} ($\beta_6$), \texttt{vacant} ($\beta_7$) and \texttt{popden} ($\beta_9$) are jointly zero.
    
$\mathcal{H}_0$:$\beta_4$=0, $\beta_6$=0, $\beta_7$=0, $\beta_9$=0\\
\ksp
\begin{knitrout}
\definecolor{shadecolor}{rgb}{1, 1, 1}\color{fgcolor}\begin{kframe}
\begin{alltt}
\hlstd{invH} \hlkwb{<-} \hlkwd{solve}\hlstd{(}\hlopt{-}\hlstd{H)}
\hlstd{Vb} \hlkwb{<-} \hlstd{invH[}\hlnum{1}\hlopt{:}\hlstd{k,} \hlnum{1}\hlopt{:}\hlstd{k]}  \hlcom{#we can use any of the ususal estimators for Vb for the Wald test;}
\hlstd{b} \hlkwb{<-} \hlstd{bm[}\hlnum{1}\hlopt{:}\hlstd{k]}  \hlcom{# here we'll stick to the inverted Hessian}
\hlstd{Rmat1} \hlkwb{<-} \hlkwd{matrix}\hlstd{(}\hlkwd{c}\hlstd{(}\hlnum{0}\hlstd{,} \hlnum{0}\hlstd{,} \hlnum{0}\hlstd{,} \hlnum{1}\hlstd{,} \hlnum{0}\hlstd{,} \hlnum{0}\hlstd{,} \hlnum{0}\hlstd{,} \hlnum{0}\hlstd{,} \hlnum{0}\hlstd{,} \hlnum{0}\hlstd{,} \hlnum{0}\hlstd{,} \hlnum{0}\hlstd{),} \hlkwc{nrow} \hlstd{=} \hlnum{1}\hlstd{)}
\hlstd{Rmat2} \hlkwb{<-} \hlkwd{matrix}\hlstd{(}\hlkwd{c}\hlstd{(}\hlnum{0}\hlstd{,} \hlnum{0}\hlstd{,} \hlnum{0}\hlstd{,} \hlnum{0}\hlstd{,} \hlnum{0}\hlstd{,} \hlnum{1}\hlstd{,} \hlnum{0}\hlstd{,} \hlnum{0}\hlstd{,} \hlnum{0}\hlstd{,} \hlnum{0}\hlstd{,} \hlnum{0}\hlstd{,} \hlnum{0}\hlstd{),} \hlkwc{nrow} \hlstd{=} \hlnum{1}\hlstd{)}
\hlstd{Rmat3} \hlkwb{<-} \hlkwd{matrix}\hlstd{(}\hlkwd{c}\hlstd{(}\hlnum{0}\hlstd{,} \hlnum{0}\hlstd{,} \hlnum{0}\hlstd{,} \hlnum{0}\hlstd{,} \hlnum{0}\hlstd{,} \hlnum{0}\hlstd{,} \hlnum{1}\hlstd{,} \hlnum{0}\hlstd{,} \hlnum{0}\hlstd{,} \hlnum{0}\hlstd{,} \hlnum{0}\hlstd{,} \hlnum{0}\hlstd{),} \hlkwc{nrow} \hlstd{=} \hlnum{1}\hlstd{)}
\hlstd{Rmat4} \hlkwb{<-} \hlkwd{matrix}\hlstd{(}\hlkwd{c}\hlstd{(}\hlnum{0}\hlstd{,} \hlnum{0}\hlstd{,} \hlnum{0}\hlstd{,} \hlnum{0}\hlstd{,} \hlnum{0}\hlstd{,} \hlnum{0}\hlstd{,} \hlnum{0}\hlstd{,} \hlnum{0}\hlstd{,} \hlnum{1}\hlstd{,} \hlnum{0}\hlstd{,} \hlnum{0}\hlstd{,} \hlnum{0}\hlstd{),} \hlkwc{nrow} \hlstd{=} \hlnum{1}\hlstd{)}
\hlstd{Rmat} \hlkwb{<-} \hlkwd{rbind}\hlstd{(Rmat1, Rmat2, Rmat3, Rmat4)}
\hlstd{q} \hlkwb{<-} \hlkwd{matrix}\hlstd{(}\hlkwd{c}\hlstd{(}\hlnum{0}\hlstd{,} \hlnum{0}\hlstd{,} \hlnum{0}\hlstd{,} \hlnum{0}\hlstd{),} \hlkwc{nrow} \hlstd{=} \hlnum{4}\hlstd{)}
\hlstd{J} \hlkwb{<-} \hlkwd{nrow}\hlstd{(Rmat)}

\hlstd{W} \hlkwb{<-} \hlkwd{t}\hlstd{(Rmat} \hlopt \hlstd{b} \hlopt{-} \hlstd{q)} \hlopt \hlkwd{solve}\hlstd{(Rmat} \hlopt \hlstd{Vb} \hlopt \hlkwd{t}\hlstd{(Rmat))} \hlopt
    \hlstd{(Rmat} \hlopt \hlstd{b} \hlopt{-} \hlstd{q)}
\hlstd{pval} \hlkwb{=} \hlnum{1} \hlopt{-} \hlkwd{pchisq}\hlstd{(W, J)}
\end{alltt}
\end{kframe}
\end{knitrout}
\ksp

The Wald-statistic for this test is 1.639.\\
The corresponding p-value is 0.802.\\

Therefore we can not reject our null-hypothesis (which is marginal effects of \texttt{age} ($\beta_4$), \texttt{pkadeq} ($\beta_6$), \texttt{vacant} ($\beta_7$) and \texttt{popden} ($\beta_9$) are jointly zero) at 1\% or 5\% level of significance. 
\ksp

    \item The effect of an additional mile away from the airport on the value of a property 
is LESS OR EQUAL TO 3 times that of an additional mile from the hazardous waste site.\\
(\emph{Hint: There are multiple ways to set up the null hypothesis. Choose the one that has a zero on the right hand side of the equation.})

$\mathcal{H}_0$:$\beta_{11}$-3*$\beta_{12}$=0
\ksp
\begin{knitrout}
\definecolor{shadecolor}{rgb}{1, 1, 1}\color{fgcolor}\begin{kframe}
\begin{alltt}
\hlstd{invH} \hlkwb{<-} \hlkwd{solve}\hlstd{(}\hlopt{-}\hlstd{H)}
\hlstd{Vb} \hlkwb{<-} \hlstd{invH[}\hlnum{1}\hlopt{:}\hlstd{k,} \hlnum{1}\hlopt{:}\hlstd{k]}
\hlstd{b} \hlkwb{<-} \hlstd{bm[}\hlnum{1}\hlopt{:}\hlstd{k]}

\hlstd{Rmat1} \hlkwb{<-} \hlkwd{matrix}\hlstd{(}\hlkwd{c}\hlstd{(}\hlnum{0}\hlstd{,} \hlnum{0}\hlstd{,} \hlnum{0}\hlstd{,} \hlnum{0}\hlstd{,} \hlnum{0}\hlstd{,} \hlnum{0}\hlstd{,} \hlnum{0}\hlstd{,} \hlnum{0}\hlstd{,} \hlnum{0}\hlstd{,} \hlnum{0}\hlstd{,} \hlnum{1}\hlstd{,} \hlopt{-}\hlnum{3}\hlstd{),} \hlkwc{nrow} \hlstd{=} \hlnum{1}\hlstd{)}
\hlstd{Rmat} \hlkwb{<-} \hlkwd{rbind}\hlstd{(Rmat1)}
\hlstd{q} \hlkwb{<-} \hlkwd{matrix}\hlstd{(}\hlkwd{c}\hlstd{(}\hlnum{0}\hlstd{),} \hlkwc{nrow} \hlstd{=} \hlnum{1}\hlstd{)}
\hlstd{J} \hlkwb{<-} \hlkwd{nrow}\hlstd{(Rmat)}

\hlstd{W} \hlkwb{<-} \hlkwd{t}\hlstd{(Rmat} \hlopt \hlstd{b} \hlopt{-} \hlstd{q)} \hlopt \hlkwd{solve}\hlstd{(Rmat} \hlopt \hlstd{Vb} \hlopt \hlkwd{t}\hlstd{(Rmat))} \hlopt
    \hlstd{(Rmat} \hlopt \hlstd{b} \hlopt{-} \hlstd{q)}
\hlstd{pval} \hlkwb{=} \hlnum{1} \hlopt{-} \hlkwd{pchisq}\hlstd{(W, J)}
\end{alltt}
\end{kframe}
\end{knitrout}
\ksp

The Wald-statistic for this test is 0.046.\\
The corresponding p-value is 0.831.\\
\ksp

This value greater than .05 (5\% level of significance); so, we cannot reject our hypothesis ($\mathcal{H}_0$: $\beta_{11} - 3\beta_{12} \geq 0$). Hence it can be said that an additional mile away from the airport increases the value of a property by less than 3 times an additional mile from the hazardous waste site.  \\




    \item The ratio of (additional mile from airport / additional mile from haz. waste site) 
is NO SMALLER THAN the squared effect of "log of square footage".\\
(\emph{Hint: There are multiple ways to set up the null hypothesis. Choose the one that has a zero on the right hand side of the equation.})

$\mathcal{H}_0$: $\frac{\beta_{11}}{\beta_{12}}-(\beta_3)^2=0$\\
\ksp

\begin{knitrout}
\definecolor{shadecolor}{rgb}{1, 1, 1}\color{fgcolor}\begin{kframe}
\begin{alltt}
\hlstd{b3} \hlkwb{<-} \hlstd{b[}\hlnum{3}\hlstd{]}
\hlstd{b11} \hlkwb{=} \hlstd{b[}\hlnum{11}\hlstd{]}
\hlstd{b12} \hlkwb{=} \hlstd{b[}\hlnum{12}\hlstd{]}
\hlstd{cb} \hlkwb{=} \hlstd{(b11}\hlopt{/}\hlstd{b12)} \hlopt{-} \hlstd{(b3)}\hlopt{^}\hlstd{(}\hlnum{2}\hlstd{)}
\hlstd{q} \hlkwb{=} \hlnum{0}
\hlstd{J} \hlkwb{=} \hlnum{1}
\hlstd{delb3} \hlkwb{=} \hlopt{-}\hlstd{(}\hlnum{2} \hlopt{*} \hlstd{b3)}
\hlstd{delb11} \hlkwb{=} \hlnum{1}\hlopt{/}\hlstd{b12}
\hlstd{delb12} \hlkwb{=} \hlopt{-}\hlstd{b11}\hlopt{/}\hlstd{(b12)}\hlopt{^}\hlstd{(}\hlnum{2}\hlstd{)}
\hlstd{C} \hlkwb{<-} \hlkwd{matrix}\hlstd{(}\hlkwd{c}\hlstd{(}\hlnum{0}\hlstd{,} \hlnum{0}\hlstd{, delb3,} \hlnum{0}\hlstd{,} \hlnum{0}\hlstd{,} \hlnum{0}\hlstd{,} \hlnum{0}\hlstd{,} \hlnum{0}\hlstd{,} \hlnum{0}\hlstd{,} \hlnum{0}\hlstd{, delb11, delb12),}
    \hlkwc{nrow} \hlstd{=} \hlnum{1}\hlstd{)}
\hlstd{Vcb} \hlkwb{=} \hlstd{C} \hlopt \hlstd{Vb} \hlopt \hlkwd{t}\hlstd{(C)}
\hlstd{W} \hlkwb{=} \hlkwd{t}\hlstd{(cb} \hlopt{-} \hlstd{q)} \hlopt \hlkwd{solve}\hlstd{(Vcb)} \hlopt \hlstd{(cb} \hlopt{-} \hlstd{q)}
\hlstd{pval} \hlkwb{=} \hlnum{1} \hlopt{-} \hlkwd{pchisq}\hlstd{(W, J)}
\end{alltt}
\end{kframe}
\end{knitrout}
\ksp
\end{enumerate}
\ksp
The Wald-statistic for this test is 5.017.\\
The corresponding p-value is 0.025. \\

Based on the p-value, we can not reject the null hypothesis. Then we can say that ratio of (additional mile from airport / additional mile from hazardous waste site) is larger than the squared effect of "log of square footage.\\\\
\ksp


    
\end{enumerate}




\end{document}

 

